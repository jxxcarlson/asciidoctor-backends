
\documentclass[11pt]{amsart}
\usepackage{geometry}                % See geometry.pdf to learn the layout options. There are lots.
\geometry{letterpaper}               % ... or a4paper or a5paper or ... 
%\geometry{landscape}                % Activate for for rotated page geometry
%\usepackage[parfill]{parskip}       % Activate to begin paragraphs with an empty line rather than an indent
\usepackage{graphicx}
\usepackage{amssymb}
\usepackage{epstopdf}
\usepackage{hyperref}
\DeclareGraphicsRule{.tif}{png}{.png}{`convert #1 `dirname #1`/`basename #1 .tif`.png}

\title{A Compendius Treatise of the Mathematical Arts}
\author{Haarald Gruundahl}
\date{2/3/1601}                                        
\begin{document}
\maketitle

\parskip8pt
\parindent0pt




N
\section{Introduction}

Twas brillig and the slithey toves
went the bar and had a blast.


Ho hum, ho hum, fee fie fo fum,
said the Jolly Green Giant.




\section{Technical matters}

\subsection{Arithmetic}

$1 + 1 = 2$




\subsection{Algebra}

The name of our \textbf{much beloved} subject
comes from


\begin{itemize}

\item Italian

\item Spanish

\item medieval Latin

\item Arabic

\end{itemize}


The Arabic origin, \emph{al-jabr}, is primary.
It signifies the reunion of broken parts, bone setting,
from \emph{jabara}, reunite, restore. The original sense,
the surgical treatment of fractures, probably came
via Spanish, in which it survives; the mathematical s
ense comes from the title of a book,
ilm al-jabr wal-mukabala the science of
restoring what is missing and equating like with
like, by the mathematician al-Kwarizmi (see algorithm).


\href{http://en.wikipedia.org/wiki/History_of_algebra}{History of Algebra}




\subsection{Binomial Formula}

\[
 (a + b)^2 = a^2 + 2ab + b^2
\]


\subsubsection{Exercise}

\begin{enumerate}

\item Substitute $a = 1$, $b = 2$ in both sides
and verify that equality holds.

\item Above is the binomial formula with exponent two.
What is the binomial formul for exponent three?

\end{enumerate}


\paragraph{Comment}

We are getting so deep in the nesting that this
is getting ridiculous!


\subparagraph{Subcomment}

And even more ridiculous still!!












\section{Things to do}

Implement tables.  Such a pain in TeX, so much nicer in Asciidoc!


Also: we need to write a preamble and to
end the document properly.


PS.  We should handle both plain TeX and LaTeX.



